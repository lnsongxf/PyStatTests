% Generated by Sphinx.
\def\sphinxdocclass{report}
\documentclass[letterpaper,10pt,english]{sphinxmanual}
\usepackage[utf8]{inputenc}
\DeclareUnicodeCharacter{00A0}{\nobreakspace}
\usepackage{cmap}
\usepackage[T1]{fontenc}
\usepackage{babel}
\usepackage{times}
\usepackage[Bjarne]{fncychap}
\usepackage{longtable}
\usepackage{sphinx}
\usepackage{multirow}


\title{PyStat Documentation}
\date{July 29, 2014}
\release{0.1.1a}
\author{Tirthankar Chakravarty}
\newcommand{\sphinxlogo}{}
\renewcommand{\releasename}{Release}
\makeindex

\makeatletter
\def\PYG@reset{\let\PYG@it=\relax \let\PYG@bf=\relax%
    \let\PYG@ul=\relax \let\PYG@tc=\relax%
    \let\PYG@bc=\relax \let\PYG@ff=\relax}
\def\PYG@tok#1{\csname PYG@tok@#1\endcsname}
\def\PYG@toks#1+{\ifx\relax#1\empty\else%
    \PYG@tok{#1}\expandafter\PYG@toks\fi}
\def\PYG@do#1{\PYG@bc{\PYG@tc{\PYG@ul{%
    \PYG@it{\PYG@bf{\PYG@ff{#1}}}}}}}
\def\PYG#1#2{\PYG@reset\PYG@toks#1+\relax+\PYG@do{#2}}

\expandafter\def\csname PYG@tok@c\endcsname{\let\PYG@it=\textit\def\PYG@tc##1{\textcolor[rgb]{0.25,0.50,0.56}{##1}}}
\expandafter\def\csname PYG@tok@o\endcsname{\def\PYG@tc##1{\textcolor[rgb]{0.40,0.40,0.40}{##1}}}
\expandafter\def\csname PYG@tok@m\endcsname{\def\PYG@tc##1{\textcolor[rgb]{0.13,0.50,0.31}{##1}}}
\expandafter\def\csname PYG@tok@k\endcsname{\let\PYG@bf=\textbf\def\PYG@tc##1{\textcolor[rgb]{0.00,0.44,0.13}{##1}}}
\expandafter\def\csname PYG@tok@nt\endcsname{\let\PYG@bf=\textbf\def\PYG@tc##1{\textcolor[rgb]{0.02,0.16,0.45}{##1}}}
\expandafter\def\csname PYG@tok@w\endcsname{\def\PYG@tc##1{\textcolor[rgb]{0.73,0.73,0.73}{##1}}}
\expandafter\def\csname PYG@tok@s\endcsname{\def\PYG@tc##1{\textcolor[rgb]{0.25,0.44,0.63}{##1}}}
\expandafter\def\csname PYG@tok@bp\endcsname{\def\PYG@tc##1{\textcolor[rgb]{0.00,0.44,0.13}{##1}}}
\expandafter\def\csname PYG@tok@c1\endcsname{\let\PYG@it=\textit\def\PYG@tc##1{\textcolor[rgb]{0.25,0.50,0.56}{##1}}}
\expandafter\def\csname PYG@tok@ni\endcsname{\let\PYG@bf=\textbf\def\PYG@tc##1{\textcolor[rgb]{0.84,0.33,0.22}{##1}}}
\expandafter\def\csname PYG@tok@kd\endcsname{\let\PYG@bf=\textbf\def\PYG@tc##1{\textcolor[rgb]{0.00,0.44,0.13}{##1}}}
\expandafter\def\csname PYG@tok@il\endcsname{\def\PYG@tc##1{\textcolor[rgb]{0.13,0.50,0.31}{##1}}}
\expandafter\def\csname PYG@tok@kc\endcsname{\let\PYG@bf=\textbf\def\PYG@tc##1{\textcolor[rgb]{0.00,0.44,0.13}{##1}}}
\expandafter\def\csname PYG@tok@kn\endcsname{\let\PYG@bf=\textbf\def\PYG@tc##1{\textcolor[rgb]{0.00,0.44,0.13}{##1}}}
\expandafter\def\csname PYG@tok@sh\endcsname{\def\PYG@tc##1{\textcolor[rgb]{0.25,0.44,0.63}{##1}}}
\expandafter\def\csname PYG@tok@s2\endcsname{\def\PYG@tc##1{\textcolor[rgb]{0.25,0.44,0.63}{##1}}}
\expandafter\def\csname PYG@tok@ow\endcsname{\let\PYG@bf=\textbf\def\PYG@tc##1{\textcolor[rgb]{0.00,0.44,0.13}{##1}}}
\expandafter\def\csname PYG@tok@s1\endcsname{\def\PYG@tc##1{\textcolor[rgb]{0.25,0.44,0.63}{##1}}}
\expandafter\def\csname PYG@tok@kt\endcsname{\def\PYG@tc##1{\textcolor[rgb]{0.56,0.13,0.00}{##1}}}
\expandafter\def\csname PYG@tok@mo\endcsname{\def\PYG@tc##1{\textcolor[rgb]{0.13,0.50,0.31}{##1}}}
\expandafter\def\csname PYG@tok@kr\endcsname{\let\PYG@bf=\textbf\def\PYG@tc##1{\textcolor[rgb]{0.00,0.44,0.13}{##1}}}
\expandafter\def\csname PYG@tok@mi\endcsname{\def\PYG@tc##1{\textcolor[rgb]{0.13,0.50,0.31}{##1}}}
\expandafter\def\csname PYG@tok@vg\endcsname{\def\PYG@tc##1{\textcolor[rgb]{0.73,0.38,0.84}{##1}}}
\expandafter\def\csname PYG@tok@err\endcsname{\def\PYG@bc##1{\setlength{\fboxsep}{0pt}\fcolorbox[rgb]{1.00,0.00,0.00}{1,1,1}{\strut ##1}}}
\expandafter\def\csname PYG@tok@vi\endcsname{\def\PYG@tc##1{\textcolor[rgb]{0.73,0.38,0.84}{##1}}}
\expandafter\def\csname PYG@tok@mh\endcsname{\def\PYG@tc##1{\textcolor[rgb]{0.13,0.50,0.31}{##1}}}
\expandafter\def\csname PYG@tok@mf\endcsname{\def\PYG@tc##1{\textcolor[rgb]{0.13,0.50,0.31}{##1}}}
\expandafter\def\csname PYG@tok@sc\endcsname{\def\PYG@tc##1{\textcolor[rgb]{0.25,0.44,0.63}{##1}}}
\expandafter\def\csname PYG@tok@cm\endcsname{\let\PYG@it=\textit\def\PYG@tc##1{\textcolor[rgb]{0.25,0.50,0.56}{##1}}}
\expandafter\def\csname PYG@tok@kp\endcsname{\def\PYG@tc##1{\textcolor[rgb]{0.00,0.44,0.13}{##1}}}
\expandafter\def\csname PYG@tok@gd\endcsname{\def\PYG@tc##1{\textcolor[rgb]{0.63,0.00,0.00}{##1}}}
\expandafter\def\csname PYG@tok@ge\endcsname{\let\PYG@it=\textit}
\expandafter\def\csname PYG@tok@gh\endcsname{\let\PYG@bf=\textbf\def\PYG@tc##1{\textcolor[rgb]{0.00,0.00,0.50}{##1}}}
\expandafter\def\csname PYG@tok@gi\endcsname{\def\PYG@tc##1{\textcolor[rgb]{0.00,0.63,0.00}{##1}}}
\expandafter\def\csname PYG@tok@go\endcsname{\def\PYG@tc##1{\textcolor[rgb]{0.20,0.20,0.20}{##1}}}
\expandafter\def\csname PYG@tok@gr\endcsname{\def\PYG@tc##1{\textcolor[rgb]{1.00,0.00,0.00}{##1}}}
\expandafter\def\csname PYG@tok@gs\endcsname{\let\PYG@bf=\textbf}
\expandafter\def\csname PYG@tok@gp\endcsname{\let\PYG@bf=\textbf\def\PYG@tc##1{\textcolor[rgb]{0.78,0.36,0.04}{##1}}}
\expandafter\def\csname PYG@tok@nc\endcsname{\let\PYG@bf=\textbf\def\PYG@tc##1{\textcolor[rgb]{0.05,0.52,0.71}{##1}}}
\expandafter\def\csname PYG@tok@gt\endcsname{\def\PYG@tc##1{\textcolor[rgb]{0.00,0.27,0.87}{##1}}}
\expandafter\def\csname PYG@tok@gu\endcsname{\let\PYG@bf=\textbf\def\PYG@tc##1{\textcolor[rgb]{0.50,0.00,0.50}{##1}}}
\expandafter\def\csname PYG@tok@vc\endcsname{\def\PYG@tc##1{\textcolor[rgb]{0.73,0.38,0.84}{##1}}}
\expandafter\def\csname PYG@tok@cs\endcsname{\def\PYG@tc##1{\textcolor[rgb]{0.25,0.50,0.56}{##1}}\def\PYG@bc##1{\setlength{\fboxsep}{0pt}\colorbox[rgb]{1.00,0.94,0.94}{\strut ##1}}}
\expandafter\def\csname PYG@tok@cp\endcsname{\def\PYG@tc##1{\textcolor[rgb]{0.00,0.44,0.13}{##1}}}
\expandafter\def\csname PYG@tok@se\endcsname{\let\PYG@bf=\textbf\def\PYG@tc##1{\textcolor[rgb]{0.25,0.44,0.63}{##1}}}
\expandafter\def\csname PYG@tok@sx\endcsname{\def\PYG@tc##1{\textcolor[rgb]{0.78,0.36,0.04}{##1}}}
\expandafter\def\csname PYG@tok@ne\endcsname{\def\PYG@tc##1{\textcolor[rgb]{0.00,0.44,0.13}{##1}}}
\expandafter\def\csname PYG@tok@sr\endcsname{\def\PYG@tc##1{\textcolor[rgb]{0.14,0.33,0.53}{##1}}}
\expandafter\def\csname PYG@tok@ss\endcsname{\def\PYG@tc##1{\textcolor[rgb]{0.32,0.47,0.09}{##1}}}
\expandafter\def\csname PYG@tok@nv\endcsname{\def\PYG@tc##1{\textcolor[rgb]{0.73,0.38,0.84}{##1}}}
\expandafter\def\csname PYG@tok@nn\endcsname{\let\PYG@bf=\textbf\def\PYG@tc##1{\textcolor[rgb]{0.05,0.52,0.71}{##1}}}
\expandafter\def\csname PYG@tok@nl\endcsname{\let\PYG@bf=\textbf\def\PYG@tc##1{\textcolor[rgb]{0.00,0.13,0.44}{##1}}}
\expandafter\def\csname PYG@tok@no\endcsname{\def\PYG@tc##1{\textcolor[rgb]{0.38,0.68,0.84}{##1}}}
\expandafter\def\csname PYG@tok@si\endcsname{\let\PYG@it=\textit\def\PYG@tc##1{\textcolor[rgb]{0.44,0.63,0.82}{##1}}}
\expandafter\def\csname PYG@tok@na\endcsname{\def\PYG@tc##1{\textcolor[rgb]{0.25,0.44,0.63}{##1}}}
\expandafter\def\csname PYG@tok@sd\endcsname{\let\PYG@it=\textit\def\PYG@tc##1{\textcolor[rgb]{0.25,0.44,0.63}{##1}}}
\expandafter\def\csname PYG@tok@nb\endcsname{\def\PYG@tc##1{\textcolor[rgb]{0.00,0.44,0.13}{##1}}}
\expandafter\def\csname PYG@tok@sb\endcsname{\def\PYG@tc##1{\textcolor[rgb]{0.25,0.44,0.63}{##1}}}
\expandafter\def\csname PYG@tok@nd\endcsname{\let\PYG@bf=\textbf\def\PYG@tc##1{\textcolor[rgb]{0.33,0.33,0.33}{##1}}}
\expandafter\def\csname PYG@tok@nf\endcsname{\def\PYG@tc##1{\textcolor[rgb]{0.02,0.16,0.49}{##1}}}

\def\PYGZbs{\char`\\}
\def\PYGZus{\char`\_}
\def\PYGZob{\char`\{}
\def\PYGZcb{\char`\}}
\def\PYGZca{\char`\^}
\def\PYGZam{\char`\&}
\def\PYGZlt{\char`\<}
\def\PYGZgt{\char`\>}
\def\PYGZsh{\char`\#}
\def\PYGZpc{\char`\%}
\def\PYGZdl{\char`\$}
\def\PYGZhy{\char`\-}
\def\PYGZsq{\char`\'}
\def\PYGZdq{\char`\"}
\def\PYGZti{\char`\~}
% for compatibility with earlier versions
\def\PYGZat{@}
\def\PYGZlb{[}
\def\PYGZrb{]}
\makeatother

\begin{document}

\maketitle
\tableofcontents
\phantomsection\label{index::doc}



\chapter{Introduction}
\label{index:pystat-documentation}\label{index:introduction}
This package contains commonly used statistical routines implemented in Python.
The functionality is organized in 2 broad categories:
\begin{itemize}
\item {} 
Estimation routines

\item {} 
Testing routines

\end{itemize}

The classical, mostly parametric tests implemented in this package are taken from the


\chapter{Statistical Tests}
\label{index:module-tests.tests}\label{index:statistical-tests}\index{tests.tests (module)}\index{test\_correlation() (in module tests.tests)}

\begin{fulllineitems}
\phantomsection\label{index:tests.tests.test_correlation}\pysiglinewithargsret{\code{tests.tests.}\bfcode{test\_correlation}}{\emph{series1}, \emph{series2}, \emph{correlation}, \emph{test\_type='z'}}{}
Test population correlation is equal to a given value
if the correlation passed is zero, then a t-test is called
else, a z-test is called. Actually it is possible to also compute the
z-test when the hypothesised population correlation is zero.
\begin{quote}\begin{description}
\item[{Param}] \leavevmode
series1, series2: the pandas series whose correlation has to be computed

\item[{Returns}] \leavevmode
the test statistic, the p-value, the degrees of freedom, and the type of the test

\end{description}\end{quote}

\end{fulllineitems}

\index{test\_mean() (in module tests.tests)}

\begin{fulllineitems}
\phantomsection\label{index:tests.tests.test_mean}\pysiglinewithargsret{\code{tests.tests.}\bfcode{test\_mean}}{\emph{series}, \emph{mean}, \emph{variance=None}, \emph{type='two-sided'}}{}
Test that the mean of the series is equal to given number.
\begin{quote}\begin{description}
\item[{Param}] \leavevmode
series: a pandas Series

\item[{Param}] \leavevmode
mean: the hypothesised true mean of the population

\item[{Param}] \leavevmode
variance: if supplied, the known variance of the population

\item[{Returns}] \leavevmode
a dict with the statistic and the p-value

\end{description}\end{quote}

This module implements two tests of population mean -- one where the 
population variance is assumed to be known, and the other where the
population variance is assumed to be unknown.

\textasciicircum{} Test: Z-test for population mean (variance known)

\textbf{Data:} The data is assumed to be a random (IID) sample from a normal population \footnote{
Note that the distribution of the statistic might be robust to other
kinds of populations and sampling schemes but we discuss only the cases in {\hyperref[index:kanji98]{{[}kanji98{]}}}.
}.

\textbf{Hypothesis}:
\begin{gather}
\begin{split}H_0: \mu = \mu_0 \\ 
H_1: \mu \neq \mu_0\end{split}\notag
\end{gather}
\begin{Verbatim}[commandchars=\\\{\}]
\PYG{k+kn}{import} \PYG{n+nn}{pandas} \PYG{k+kn}{as} \PYG{n+nn}{pd}
\PYG{k+kn}{import} \PYG{n+nn}{numpy} \PYG{k+kn}{as} \PYG{n+nn}{np}
\PYG{c}{\PYGZsh{} import unittest as ut}

\PYG{k+kn}{from} \PYG{n+nn}{PyStatTests.tests} \PYG{k+kn}{import} \PYG{n}{tests}

\PYG{c}{\PYGZsh{}================================================}
\PYG{c}{\PYGZsh{} test 1: Z\PYGZhy{}test for a population mean (variance known)}
\PYG{c}{\PYGZsh{} test 7: t\PYGZhy{}test for population mean (variance unknown)}
\PYG{c}{\PYGZsh{}================================================}

\PYG{k}{print}\PYG{p}{(}\PYG{l+s}{\PYGZdq{}}\PYG{l+s}{==============================================================}\PYG{l+s}{\PYGZdq{}}\PYG{p}{)}
\PYG{k}{print}\PYG{p}{(}\PYG{l+s}{\PYGZdq{}}\PYG{l+s}{Tests of population mean under the null}\PYG{l+s}{\PYGZdq{}}\PYG{p}{)}
\PYG{k}{print}\PYG{p}{(}\PYG{l+s}{\PYGZdq{}}\PYG{l+s}{==============================================================}\PYG{l+s}{\PYGZdq{}}\PYG{p}{)}
\PYG{c}{\PYGZsh{} generate the univariate series}
\PYG{n}{series1} \PYG{o}{=} \PYG{n}{pd}\PYG{o}{.}\PYG{n}{Series}\PYG{p}{(}\PYG{n}{np}\PYG{o}{.}\PYG{n}{random}\PYG{o}{.}\PYG{n}{randn}\PYG{p}{(}\PYG{l+m+mi}{10}\PYG{p}{)}\PYG{p}{)}
\PYG{n}{help}\PYG{p}{(}\PYG{n}{tests}\PYG{o}{.}\PYG{n}{test\PYGZus{}means}\PYG{p}{)}
\PYG{c}{\PYGZsh{} test 1: null  DGP}
\PYG{k}{print}\PYG{p}{(}\PYG{n}{tests}\PYG{o}{.}\PYG{n}{test\PYGZus{}mean}\PYG{p}{(}\PYG{n}{series1}\PYG{p}{,} \PYG{n}{mean} \PYG{o}{=} \PYG{l+m+mi}{0}\PYG{p}{,} \PYG{n}{variance} \PYG{o}{=} \PYG{l+m+mi}{1}\PYG{p}{)}\PYG{p}{)}
\PYG{c}{\PYGZsh{} test 7: null DGP}
\PYG{k}{print}\PYG{p}{(}\PYG{n}{tests}\PYG{o}{.}\PYG{n}{test\PYGZus{}mean}\PYG{p}{(}\PYG{n}{series1}\PYG{p}{,} \PYG{n}{mean} \PYG{o}{=} \PYG{l+m+mi}{0}\PYG{p}{)}\PYG{p}{)}

\PYG{k}{print}\PYG{p}{(}\PYG{l+s}{\PYGZdq{}}\PYG{l+s}{==============================================================}\PYG{l+s}{\PYGZdq{}}\PYG{p}{)}
\PYG{k}{print}\PYG{p}{(}\PYG{l+s}{\PYGZdq{}}\PYG{l+s}{Tests of population mean under the alternative}\PYG{l+s}{\PYGZdq{}}\PYG{p}{)}
\PYG{k}{print}\PYG{p}{(}\PYG{l+s}{\PYGZdq{}}\PYG{l+s}{==============================================================}\PYG{l+s}{\PYGZdq{}}\PYG{p}{)}
\PYG{c}{\PYGZsh{} test 1: alternative DGP}
\PYG{k}{print}\PYG{p}{(}\PYG{n}{tests}\PYG{o}{.}\PYG{n}{test\PYGZus{}mean}\PYG{p}{(}\PYG{n}{series1}\PYG{p}{,} \PYG{n}{mean} \PYG{o}{=} \PYG{l+m+mi}{2}\PYG{p}{,} \PYG{n}{variance} \PYG{o}{=} \PYG{l+m+mi}{1}\PYG{p}{)}\PYG{p}{)}
\PYG{c}{\PYGZsh{} test 7: alternative DGP}
\PYG{k}{print}\PYG{p}{(}\PYG{n}{tests}\PYG{o}{.}\PYG{n}{test\PYGZus{}mean}\PYG{p}{(}\PYG{n}{series1}\PYG{p}{,} \PYG{n}{mean} \PYG{o}{=} \PYG{l+m+mi}{2}\PYG{p}{)}\PYG{p}{)}
\end{Verbatim}

\end{fulllineitems}

\index{test\_means() (in module tests.tests)}

\begin{fulllineitems}
\phantomsection\label{index:tests.tests.test_means}\pysiglinewithargsret{\code{tests.tests.}\bfcode{test\_means}}{\emph{series1}, \emph{series2}, \emph{variance1=None}, \emph{variance2=None}, \emph{var\_equal=False}, \emph{var\_unknown=False}, \emph{type='two-sided'}}{}
Test whether two population means are different
\begin{quote}\begin{description}
\item[{Param}] \leavevmode
series1, series 2: two pandas series from the two populations to be compared

\item[{Param}] \leavevmode
variance1, variance2: if supplied, the known variances of the populations

\item[{Param}] \leavevmode
var\_equal: flag for if the two variances are equal

\item[{Param}] \leavevmode
var\_unknown: flag for whether the variances are unknown

\item[{Param}] \leavevmode
type: whether to computer a two-sided, upper or lower tailed test

\item[{Returns}] \leavevmode
a dict with the statistic, the p-value and the degrees of freedom of the test

\end{description}\end{quote}

\end{fulllineitems}

\index{test\_proportion() (in module tests.tests)}

\begin{fulllineitems}
\phantomsection\label{index:tests.tests.test_proportion}\pysiglinewithargsret{\code{tests.tests.}\bfcode{test\_proportion}}{\emph{series}, \emph{proportion}}{}
Test whether the population proportion is equal to a given value
\begin{quote}\begin{description}
\item[{Param}] \leavevmode
series: sample of data from a population

\item[{Param}] \leavevmode
proportion: hypothesised proportion of the binomial population

\item[{Returns}] \leavevmode
a dict containing the test statistic and the p-value of the test

\end{description}\end{quote}

\end{fulllineitems}

\index{test\_proportions() (in module tests.tests)}

\begin{fulllineitems}
\phantomsection\label{index:tests.tests.test_proportions}\pysiglinewithargsret{\code{tests.tests.}\bfcode{test\_proportions}}{\emph{series1}, \emph{series2}}{}
Test the significance of the difference between two proportions
\begin{quote}\begin{description}
\item[{Param}] \leavevmode
series1, series2: pandas series from which the proportion is to be computed

\item[{Returns}] \leavevmode
a dict with the statistic and the p-value

\end{description}\end{quote}

\end{fulllineitems}

\index{test\_variance() (in module tests.tests)}

\begin{fulllineitems}
\phantomsection\label{index:tests.tests.test_variance}\pysiglinewithargsret{\code{tests.tests.}\bfcode{test\_variance}}{\emph{series}, \emph{variance}}{}
Test whether the population variance is equal to a given value
\begin{quote}\begin{description}
\item[{Param}] \leavevmode
series: sample of data from a population

\item[{Param}] \leavevmode
variance: hypothesised variance of the population

\item[{Returns}] \leavevmode
a dict containing the statistic, p-value and the degrees of freedom of the test

\end{description}\end{quote}

\end{fulllineitems}



\chapter{Indices and tables}
\label{index:indices-and-tables}\begin{itemize}
\item {} 
\emph{genindex}

\item {} 
\emph{modindex}

\item {} 
\emph{search}

\end{itemize}


\chapter{Bibliography}
\label{index:bibliography}
\begin{thebibliography}{kanji98}
\bibitem[kanji98]{kanji98}{\phantomsection\label{index:kanji98} 
Kanji, Gopal K. 100 statistical tests. Sage, 2006.
}
\end{thebibliography}


\renewcommand{\indexname}{Python Module Index}
\begin{theindex}
\def\bigletter#1{{\Large\sffamily#1}\nopagebreak\vspace{1mm}}
\bigletter{t}
\item {\texttt{tests.tests}}, \pageref{index:module-tests.tests}
\end{theindex}

\renewcommand{\indexname}{Index}
\printindex
\end{document}
